\documentclass{article}
\usepackage{amsmath}
\usepackage{amssymb}
\usepackage{algorithm}
\usepackage{float}
\usepackage{color}
\usepackage{multicol}
\usepackage{forloop}
\usepackage{graphicx}
\usepackage[margin=0.8in]{geometry}
\usepackage{caption}
\usepackage{enumerate}
\graphicspath{ {.} }
\title{STAT 2509B4 \\
	\large{Assignment 1}}
\author{Krystian Wojcicki, 101001444}
\date{Winter 2020}

\begin{document}
\maketitle

\begin{enumerate}[1.]
\item

\textbf{Athletes are constantly seeking measures of the degree of their cardiovascular fitness prior
to a major race. Athletes want to know when their training is at a level that will produce a
peak performance. One such measure of fitness is the time to exhaustion from running on a
treadmill at a specified angle and speed. The important question is then “Does this measure
of cardiovascular fitness translate into performance in a 10-km running race?” Twenty experienced distance runners who professed to be at top condition were evaluated on the treadmill
and then had their times recorded in a 10-km race. The data are given in the table below.:\\
\begin{center}
 \begin{tabular}{||c c c c||} 
 \hline
 Col1 & Col2 & Col2 & Col3 \\ [0.5ex] 
 \hline\hline
 1 & 6 & 87837 & 787 \\ 
 \hline
 2 & 7 & 78 & 5415 \\
 \hline
 3 & 545 & 778 & 7507 \\
 \hline
 4 & 545 & 18744 & 7560 \\
 \hline
 5 & 88 & 788 & 6344 \\ [1ex] 
 \hline
\end{tabular}
\end{center}
}
\begin{enumerate}[(a)]
  \item \textbf{Draw a scatter plot (using SAS, see part (i)) to get an idea of the form of the relationship
between the treadmill time (x) and 10-km running time (y). Does the scatter plot suggest
an approximate linear relationship between the two variables?: } a characteristic that varies from a person or a thing to another one, or over time.
  \item \textbf{State a simple linear regression (SLR) model for two variables and describe all assumptions that are necessary for statistical inference.
: } \\
Model $y = \beta_0 + \beta_1 * x+ \epsilon, n = 20$.
Assumptions
  \begin{enumerate}[(1)]
   \item The random errors $\epsilon_i$'s are mutually independent.
   \item $\epsilon_i$'s are normally distributed
   \item $\epsilon_i$'s have common mean 0 in other words $E(\epsilon_i) = 0$ for all $i$.
   \item $\epsilon_i$'s have common variance $\sigma^2$ meaning $Var(\epsilon_i) = \sigma^2)$ for all $i$
  \item $x$'s are observed without error.
\end{enumerate}
  \item \textbf{Find the least squares estimates of $\beta_0$ and $\beta_1$ in the SLR model. Find the least square
fitted regression line.: } \\
\begingroup
\Large
\begin{equation}
\hat{\beta_1} = \frac{S_{xy}}{S_{xx}} = \frac{  \sum_{i=1}^{n}{x_iy_i }  - \frac{  \sum_{i=1}^{n}{x_i}  \sum_{i=1}^{n}{y_i}  }{n} }
{  \sum_{i=1}^{n}{x_{i}^2} - \frac{    (\sum_{i=1}^{n}{x_i})^2    }{ n }}   = \frac{ 7852.25 - \frac{195.1 * 812}{20}  }{ 1940.05 - \frac{(195.1)^2 }{20} }  = -1.8673252 \simeq -1.87  \nonumber
\end{equation}
$\hat{\beta_0} = \bar{y} - \hat{\beta_1}\bar{x} = \frac{ \sum_{i=1}^{n}{y_i} }{n} - \hat{\beta_1}\frac{\sum_{i=1}^{n}{x_i} }{n} = \frac{812}{20} + 1.87 * \frac{195.1}{20} = 58.815757 \simeq 58.82 $
\endgroup
Therefore the least square fitted regression is given by $\hat{y} = 58.82 - 1.87x$ 

\item \textbf{Find $s^2$, an estimate of $\sigma^2$: } \\

\begingroup
\Large
$s^2 = \frac{SSE}{n-2} = \frac{S_{yy} - \frac{S_{xy}^2}{S_{xx}}}{n-2} = \frac{            (\sum_{i=1}^{n}{y_i^2} - \frac{ (\sum_{i=1}^{n}{y_i})^2}{n} )  - 
\frac{   ( \sum_{i=1}^{n}{x_iy_i }  - \frac{  \sum_{i=1}^{n}{x_i}  \sum_{i=1}^{n}{y_i}  }{n})^2     }
{    (\sum_{i=1}^{n}{x_i^2} - \frac{ (\sum_{i=1}^{n}{x_i})^2}{n} )     }}
{n - 2}$ \\
$ = \frac{ (33175.2 - \frac{812^2}{20}) - \frac{ (7852.25 - \frac{195.1 * 812}{20})^2}{1940.05-\frac{195.1^2}{20}}  }{18} = 4.41718627269 \simeq 4.42 $
\endgroup

Therefore $s = \sqrt{s^2} = \sqrt{4.42} = 2.10171032083 \simeq 2.10$

  \item \textbf{Use the t-test to test whether there is a significant linear relationship between 10-km
running time and the treadmill time. Use $\alpha$ = 0.05.: }

$H_0: \beta_1 = 0, H_a: B_1 \neq 0$ \\
$\alpha = 0.05 \to \alpha/2 = 0.025$

Since we are using a t-test, our test statistic is t and $t= \frac{\hat{\beta_1} - 0}{s/\sqrt{S_{xx}}} = \frac{-1.87}{2.10/\sqrt{36.8495}} = -5.3934 \simeq -5.39$

Rejection region, we reject $H_0$ if $|t| > t_{\alpha/2;n-2} = t_{0.025;18} = 2.101$

Since $|t| = |-5.39| = 5.39 > 2.101$, we reject $H_0$ and we can conclude that at $\alpha = 0.05$ or 5\% level there is evidence that there is a linear relationship between 10-km running time and the treadmill time.

  \item \textbf{ Find a 95\% confidence interval for $\beta_1$.: }

$1 - \alpha = 0.95 \to \alpha = 0.05 \to \alpha/2 = 0.025$

Therefore $\beta_1$'s 95\% confidence interval is $(\hat{\beta_1} \pm t_{\alpha/2;n-2}\frac{s}{\sqrt{S_{xx}}}) = (-1.87 \pm 2.101 * \frac{2.10}{\sqrt{36.85}}) = (-2.59474163696, -1.13990876535) \simeq (-2.60, -1.14) $. And we can be 95\% confident that the true value of $\beta_1$ lies in the interval $(-2.60, -1.14)$.

  \item \textbf{ Set up the ANOVA table and use it to test whether there is a significant linear relationship between 10-km running time and the treadmill time. Use $\alpha$ = 0.05.: }

$TSS = S_{yy} = \sum_{i=1}^{n}{y_i^2} - \frac{ (\sum_{i=1}^{n}{y_i})^2 }{n} = 33175.2 - \frac{812^2}{20} = 208 $ \\
$SSR = \frac{S_{xy}^2}{S_{xx}} = \frac{ (7852.25 - \frac{195.1 * 812}{20})^2}{ 1940.05 - \frac{195.1^2}{20}} = 128.49 $ \\
$SSE = TSS - SSR = 208 - 128.49 = 79.51 $ \\ 
$MSR = SSR/1 = 128.49 $ \\ 
$MSE = \frac{SSE}{n-2} = \frac{79.51}{18} = 4.42$ \\
$F = \frac{MSR}{MSE} = \frac{128.49}{4.42} =  29.09 $

TODO: table

$H_0: \beta_1 = 0, H_a: \beta_1 \neq 0$
With $\alpha = 0.05$.

Using F-test so statistic is $F - \frac{MSR}{MSE} = 29.09$

Rejection region, we reject $H_0$ if $F > F_{1,n-2;\alpha} = F_{1,18;0.05} = 4.41$.

Since $F = 29.09 > 4.41$ we can reject $H_0$ and conclude that at a 5\% level of significance there is evidence of a linear relationship between the 10-km running time and the treadmill time. 

  \item \textbf{ Find the values of the coefficient of correlation, r, and the coefficient of determination,$r^2$, and interpret their meaning in this problem.: }

$r = \frac{S_{xy}}{\sqrt{S_{xx}S_{yy}}} = \frac{7852.25-\frac{195.1*812}{20}}{\sqrt{ (1940.05 - \frac{195.1^2}{20}) * (33175.2-\frac{812^2}{20})}} = -0.785966599565 \simeq -0.79$

$r^2 = \frac{SSR}{TSS} = \frac{128.49}{208} = 0.617740384615 \simeq 0.62$

  \item \textbf{Verify your results for (b) to (h) using SAS.}

\end{enumerate}

\item \textbf{Refer to Question 1.}
\begin{enumerate}[(a)]
\item \textbf{Find a 95\% confidence interval for the mean
value of the response variable (i.e. the 10-km running time) and a 95\% prediction
interval for an individual value of the response variable when the treadmill time is 9.5
minutes. What can you say about the widths of these two intervals.:}


95\% confidence interval for $E(y)$ when $x_p =9.5$. 

$\hat{y} = 58.82 - 1.87(9.5) = 41.055$ and since $1 - \alpha \to 0.95 \to \alpha = 0.05 \to \alpha/2 = 0.025$

Therefore $E(9.5)$ falls into the interval $(\hat{y} \pm t_{\alpha/2, n-2} * s * \sqrt{ \frac{1}{n} + \frac{(x_p - \bar{x})^2}{S_{xx}}}) = (41.06 \pm 2.101 * 2.10 * \sqrt{\frac{1}{20} + \frac{(41.06 - 9.755)^2}{ 1940.05 - \frac{195.1^2}{20}} }) = (39.9953486251, 42.0646513749)$

Therefore $E(9.5)$ falls into the interval $(\hat{y} \pm t_{\alpha/2, n-2} * s * \sqrt{ 1+ \frac{1}{n} + \frac{(x_p - \bar{x})^2}{S_{xx}}}) = (41.06 \pm 2.101 * 2.10 * \sqrt{1 + \frac{1}{20} + \frac{(41.06 - 9.755)^2}{ 1940.05 - \frac{195.1^2}{20}} }) = (36.4714602301, 45.5885397699)$

\end{enumerate}
\end{enumerate}
\end{document}